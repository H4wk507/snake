\documentclass[12pt]{article}
\usepackage{polski}
\usepackage[utf8]{inputenc}
\usepackage{graphicx}
\usepackage[normalem]{ulem}
\usepackage{listings}
\usepackage{hyperref}

\title{%
    Snake \\
    \large Programowanie I - projekt zaliczeniowy}

\author{Bartłomiej Pacia}
\date{Styczeń 2022}

\begin{document}

\maketitle

\section{Introduction}

“Snake” jest klonem popularnej “gry w węża”. Rozgrywka toczy się na kwadratowej
planszy. Na początku gracz startuje z wężem o długości 1 (tzn. zajmującym 1
kratkę). Jednocześnie w losowej kratce na mapie pojawia się punkt. Zadaniem
gracza jest takie kierowanie swoim wężem przy użyciu klawiszy strzałek na
klawiaturze, by zebrać punkt. Zebrany punkt znika, a długość węża gracza
zwiększa się o 1. Następnie pojawia się nowy punkt i cały proces zaczyna się od
nowa. Gracz przegra, jeśli uderzy głową węża w jego własne ciało lub wyjdzie
poza mapę. Uniknięcie tej pierwszej sytuacji staje się coraz trudniejsze wraz ze
wzrostem długości węża.

\section{Wymagania}

\sout{Przekreślone} elementy listy oznaczają wymagania zdefiniowane w
pierwotnych Założeniach do Projektu, które ostatecznie nie zostały zrealizowane.
\textbf{Pogrubione} elementy listy oznaczają nowe wymagania, niezdefiniowane w
pierwotnych Założeniach do Projektu, które zostały zrealizowane.
\begin{itemize}
    \item \sout{ekran menu z 2 przyciskami (“Graj” i “Wyjdź”)}

    \item ekran gry z \sout{2 przyciskami (“Pauza” i “Wyjdź”),} licznikiem
          obecnej liczby punktów oraz najwyższej liczby punktów

    \item zbieranie punktów i w konsekwencji wydłużanie się węża

    \item zapisywanie najwyższej liczby punktów (high score) do pliku i
          możliwość pobicia tego rekordu

    \item \textbf{możliwość podania rozmiaru planszy i interwału czasowego
              między ruchami jako argumenty do programu}

\end{itemize}

\section{Przebieg realizacji}

% Wykonawca opisuje wykonane przez siebie zadania. Należy zamieścić opis plików
% z których składa się projekt, opis algorytmu, gdy program jest związany z
% algorytmiką. W przypadku korzystania z zewnętrznych bibliotek należy je tu
% krótko opisać (do czego służą, z jakich funkcji się korzystało)

Starałem pisać się kod z użyciem nowych funkcjonalności zapewnianych przez
standardy C++11, C++14 i C++17. W szczególności użyłem \textit{smart pointers}
zamiast \textit{raw pointers} oraz przekazywania argumentów do funkcji i metod
przez referencję. Użyłem też kontenera \textit{std::vector} oraz funkcji z
biblioteki \textit{algorithm}.

\subsection{Użyte biblioteki}

Użyłem biblioteki standardowej C++ oraz biblioteki SFML, która udostępnia
podstawowe funkcje do obsługi grafiki.

\subsection{Użyte narzędzia}

Użyłem systemu kontroli wersji \textit{git} oraz serwisu \textit{GitHub} do
hostowania repozytorium projektu.

Użyłem też narzędzia \textit{clang-format}, aby utrzymywać spójny styl kodu.
Użyty przeze mnie styl to \textit{Chromium}.

\subsection{Skomplikowanie}
W trakcie implementacji okazało się, że występuje sporo przypadków krańcowych, o
których nie pomyślałem na początku.

Przykładowy taki przypadek występuje, gdy wąż zbierze punkt i jego długość
zwiększa się. Gdzie wtedy powinien pojawić się nowy fragment węża?

Od dawna nie zajmowałem się pisaniem gier, i przypomniało mi się, jak
niesamowicie szybko rośnie poziom skomplikowania kodu gry, szczególnie gdy pisze
go niedoświadczona osoba, taka jak ja. Z powodu ograniczonego czasu oraz wiedzy
(i ograniczonego czasu na jej uzupełnienie) nie użyłem żadnych wzorców
projektowych, które na pewno pomogłyby mi lepiej zdefiniować zależności między
różnymi elementami gry i w konsekwencji zmniejszyć liczbę błędów oraz czasu
wymaganego na testowanie. Jestem świadom bardzo sporego miejsca na poprawę.

\section{Instrukcja użytkownika}

\subsection{Budowanie projektu}
Aby skompilować projekt, potrzebny jest:
\begin{itemize}
    \item kompilator C++, np. \textit{g++} lub \textit{clang++}

    \item program \textit{make}

    \item źródła biblioteki SFML, którą można zainstalować stworzonym przeze
          mnie skryptem \textit{install\_sfml} dostępnym w projekcie
\end{itemize}

Po spełnieniu powyższych wymagań możemy w głównym katalogu projektu uruchomić
program \textit{make}


\begin{lstlisting}
$ make snake
\end{lstlisting}

\subsection{Uruchamianie gry}

Grę należy uruchamiać z linii komend. Pozwala to na wyświetlanie pomocy oraz
przekazanie mu argumentów wpływających na wygląd i gameplay.

Aby uruchomić grę z domyślnymi ustawieniami (ruch węża co 500 ms i plansza o
rozmiarze 16 x 16 kratek), nie trzeba podawać żadnych argumentów.


\begin{lstlisting}
$ ./snake
\end{lstlisting}

Aby uruchomić grę z wężem poruszającym się co 100 ms i planszą o rozmiarze 128 x
128 kratek, należy podać następujące argumenty:

\begin{lstlisting}
$ ./snake --interval 100 --grid-size 128
\end{lstlisting}

Aby poruszać wężem należy używać strzałek.

\section{Podsumowanie i wnioski}
% W miejscu tym piszemy co zrealizowaliśmy, z czym były problemy. Ewentualnie
% jakie są dalsze kierunki rozwoju programu, czego nie udało się zrealizować

Z ważniejszych rzeczy, których nie zaimplementowałem, trzeba wymienić interfejs
użytkownika, np. w postaci menu.

Kiedy nabędę więcej doświadczenia z tzw. nowoczesnym C++ oraz przeczytam książkę
Game Programming Patterns, planuję zrefaktoryzować kod gry, tak by nie był
zagmatwaną plątaniną wzajemnych, nieoczywistych zależności i zachowań pomiędzy
komponentami.

Chciałbym również zainteresować się tematem testów jednostkowych w C++ z użyciem
biblioteki Google Test, i (gdy już zrefaktoryzuję kod i będzie on możliwy do
testowania), planuję takowe napisać. Sprawi to, że będę pewniejszy tego, że gra
działa poprawnie i nie będę testować wszystkich jej funkcji po wprowadzeniu
jakiejś zmiany.

Mam nadzieję rozwijać dalej grę, tak by nadawała się na również jako projekt
zaliczeniowy na przedmiot Programowanie II.


Repozytorium projektu na GitHubie:
\href{https://github.com/bartekpacia/snake}{github.com/bartekpacia/snake}

\end{document}
